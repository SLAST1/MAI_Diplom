\section{Результаты кластеризации и их влияние на точность\linebreak \text{восстановления} текста}

\begin{table}[ht]
\newrobustcmd{\B}{\bfseries}
\renewcommand\tabularxcolumn[1]{m{#1}}
\begin{tabularx}{1\textwidth}
{ 
  | >{\hsize=4em\raggedright\arraybackslash}X
  ||>{\centering\arraybackslash}X
  | >{\centering\arraybackslash}X
  ||>{\centering\arraybackslash}X
  |>{\centering\arraybackslash}X|}
    \hline
        & Adjusted rand score & Homogeneity Score & Медианная точность &  Максимальная точность  \\ \hline\hline
        MFCC                  &    0.317 &    0.766  &  0.720  &   0.756  \\ \hline
        MFCC + UMAP           &    0.426 &    0.932  &  0.890  &   0.901 \\ \hline
        MFCC + UMAP + CBoS    & \B 0.481 & \B 0.937  &\B0.895  &\B 0.905  \\ \hline
\end{tabularx}
\captionof{table}{Сравнение качества восстановления текста для различных наборов признаков}
\label{tab:text_from_clussters}
\end{table}

\section{Финальные результаты восстановления текста на каждом\linebreak \text{этапе}}

\begin{table}[ht]
\newrobustcmd{\B}{\bfseries}
\renewcommand\tabularxcolumn[1]{m{#1}}
\begin{tabularx}{1\textwidth}
{ 
  |  >{\raggedright\arraybackslash\hspace{0pt}}X 
  | >{\centering\arraybackslash\hspace{0pt}}X
  ||>{\centering\arraybackslash\hspace{0pt}}X
  | >{\centering\arraybackslash\hspace{0pt}}X|}
    \hline
        \multicolumn{2}{|c||}{}
        &   Наши результаты & Результаты оригинальной работы \\ \hline\hline
        \multirow{3}{6em}{Восстановление из кластеров}   
                                            & Псевдослучайный текст  & \B0.902  &   0.762  \\ \cline{2-4}
                                            & Лингвистическая модель & \B0.912  &   0.872  \\ \hline \hline
        \multirow{3}{6em}{Цикл классификации}   
                                            & Псевдослучайный текст  & \B0.965  &   0.924  \\ \cline{2-4}
                                            & Лингвистическая модель &   0.955  & \B0.963  \\ \hline
\end{tabularx}
\captionof{table}{Сравнение посимвольной точности восстановления текста с оригинальной работой}
\label{tab:final_text_accuracy}
\end{table}


\subsection{Сравнение различных алгоритмов классификации}
\label{section:classifiers comparation}
В таблице \ref{tab:classifiers comparation}.

\begin{table}[ht]
\renewcommand\tabularxcolumn[1]{m{#1}}
\newrobustcmd{\B}{\bfseries}
\begin{tabularx}{1\textwidth}
{ 
  | >{\raggedright\arraybackslash}X 
  ||>{\centering\arraybackslash}X
  | >{\centering\arraybackslash}X
  |>{\centering\arraybackslash}X|}
    \hline
                    & Линейный дискриминантный анализ & Случайный лес  & Логистическая регрессия \\ \hline\hline
        Первый проход    &    0.928                     & \B 0.944     &  0.935  \\ \hline
        Второй проход    &    0.931                     & \B 0.962     &  0.937  \\ \hline
        Третий проход    &    0.928                     & \B 0.963     &  0.937  \\ \hline
\end{tabularx}
\captionof{table}{Сравнение качества восстановления текста при использовании различных алгоритмов классификации}
\label{tab:classifiers comparation}
\end{table}

\section{Сравнение качества восстановления текста с различными \linebreak \text{условиями} записи звука}

\begin{table}[ht]
\renewcommand\tabularxcolumn[1]{m{#1}}
\newrobustcmd{\B}{\bfseries}
\begin{tabularx}{1\textwidth}
{ 
  |>{\raggedright\arraybackslash}X 
  ||>{\centering\arraybackslash}X
  | >{\centering\arraybackslash}X
  |>{\centering\arraybackslash}X|}
    \hline
                                       & 1-я установка & 2-я установка & 3-я установка \\ \hline\hline
        Восстановление из кластеров    &    0.902      &    0.809        &  0.7        \\ \hline
        Цикл классификации             & \B 0.965      &    0.936        &  0.866      \\ \hline
\end{tabularx}
\centering
\captionof{table}{Сравнение качества восстановления псевдослучайного текста с различными условиями записи звука}
\label{tab:keyboards comparation}
\end{table}

\section{Ограничения настоящего подхода и дальнейшие исследования}
\label{section:limitations}
Ковычки в \textquote{полевых} условиях.
